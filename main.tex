\documentclass[3p,sort&compress]{elsarticle}
\usepackage[USenglish]{babel}
\usepackage{amsmath}
\usepackage{amssymb}
\usepackage{ragged2e}
\usepackage{amsthm}
\usepackage{MnSymbol}
\usepackage{bbm}
\usepackage[titletoc,toc,title]{appendix}
\usepackage{natbib}
\usepackage{subfig}
\usepackage{graphicx}
\usepackage{siunitx}
\usepackage{lineno}
\usepackage{hyperref}
\usepackage{todonotes}
\usepackage{proba}
\usepackage{cleveref}
\usepackage{booktabs}
\usepackage{subfig}%
\newtheorem{definition}{Definition}
\newtheorem{theorem}{Theorem}[section]
\newtheorem{corollary}{Corollary}%[theorem]
\newtheorem{lemma}[theorem]{Lemma}
\newtheorem*{remark}{Remark}
\newtheorem{assumption}{Assumption}
\providecommand{\abs}[1]{\lvert#1\rvert}
\providecommand{\norm}[1]{\lVert#1\rVert}
\DeclareRobustCommand{\1}[1]{\ensuremath \mathbbm{1}_{\{#1\}}}
\journal{Mathematical Biosciences}
%%%%%%%%%%%%%%%%%%%%%%%%%%%%%%%%%%%%%%%%%%%%%%%%%%%%%%%%%%%


\modulolinenumbers[1]
\begin{document}
    \begin{frontmatter}
        \title{
            Optimal Vaccination control in Leptospiriosis
        }
        %%%%%%%%%%%%%%%%%%%%%%%%%%%%%%%%%%%%%%%%%%%%%%%%%%%%%%%%%%%%%%%%%%%%%%%%%
        \author[add:unison]{%
            David Baca Carrasco
        }%
        %
        \author[add:conacyt_unison]{%
            Sa\'ul D\'iaz-Infante
            \corref{corresponding_author}
        }%
        \ead{
            saul.diazinfante@unison.mx
        }
        \author[add:unison]{%
            Manuel Adrian Acu\~na-Zegarra
        }%
    \ead{m.adrian.a.z@gmail.com}

    \address[add:conacyt_unison]{
        CONACYT-Universidad de Sonora, 
        Departamento de Matem\'aticas, Boulevard Luis Encinas y 
        Rosales S/N, 83000, Hermosillo, Sonora, M\'exico.
    }
    \address[add:itson]{
         ITSON's address
    }
        \cortext[corresponding_author]{Corresponding author}
        \begin{abstract}
            We survey some theoretical results about a family of optimal 
            control problems that arise in epidemiology. We also implement the
            so-called forward-backward-sweep method in Python to find approximate
            optimal control policies via the Pontryagin maximum principle. 
            In addition, four specific models are described and simulated.
        \end{abstract}
        \begin{keyword}
            Multi-host, persistence, extinction, stochastic perturbation,
            vector transmission.
        \end{keyword}
    \end{frontmatter}
    \linenumbers
%
%
\section{Introduction}
 

\section{The uncontrolled leptospirosis model}
      \begin{equation}
        \begin{aligned}
                \lambda_H &:= \frac{\beta_H B}{K + B}, \qquad
                \lambda_A := \frac{\beta_A B}{K + B} + \beta_{AA} I_A
                \\
                \frac{d S_H}{dt} &=
                    b (S_A + (1 - q) I_A + R_A)
                    - \lambda_H S_H - (\mu_H + m) S_H 
                \\
                \frac{d I_H}{dt} &=
                    (1 - \sigma)   q   b   I_A
                    + \lambda_H S_H - (\mu_H + \delta_H + \gamma_H)I_H
                \\
                \frac{d R_H}{dt} &=
                    \gamma_H I_H - (\mu_H + m) R_H
                \\
                \frac{dS_A}{dt} &=
                    \Lambda_A + m S_H - \lambda_A S_A - \gamma_A dS_A
                \\
                \frac{d I_A}{dt} &=
                    \lambda_A S_A - (\gamma_H + \gamma_A)   I_A
                \\
                \frac{dR_A}{dt} &=
                    m R_H + \gamma_A I_A - \gamma_A R_A
                \\
                \frac{dB}{dr} &=
                    \theta_H I_H + \theta_A I_A - k B
           \end{aligned}
      \end{equation}

We also consider a simplified version of the above model.
    \begin{equation}\label{main_system}
        \begin{aligned}
            \lambda_H &:= \bar{\beta}_{HH} I_H + \bar{\beta}_{AH} I_A,
            \\
            \lambda_A &:= \bar{\beta}_{HA} I_H + \bar{\beta}_{AA} I_A
                + \beta_{AA} I_A,
        \\
            \frac{d S_H}{dt} &=
            b (S_A + (1 - q) I_A + R_A)
            -\lambda_H S_H - (\mu_H + m) S_H,
        \\
            \frac{d I_H}{dt} &=
            (1 - \sigma)  q  b  I_A
            + \lambda_H S_H - (\mu_H + \delta_{H} + \gamma_H)I_H,
        \\
            \frac{d R_H}{dt} &=
            \gamma_H I_H - (\mu_H + m) R_H,
        \\
            \frac{dS_A}{dt} &=
            \Lambda_A + m S_H - \lambda_A S_A - \mu_A S_A,
        \\
            \frac{d I_A}{dt} &=
            \lambda_A S_A - (\mu_A + \gamma_A)   I_A,
        \\
            \frac{dR_A}{dt} &=
            m R_H + \gamma_A I_A - \mu_A R_A.
    \end{aligned}
\end{equation}

    \section{Control policies for Leptospirosis in Livestock}
    \paragraph{Notation}
    We denote by $u_1(t)$ the proportion of vaccinated livestock,
    and by $u_2(t)$ the bacterial fumigation strategies....
    Our controlled model reads:
    \begin{equation}\label{controlled_system}
    \begin{aligned}
        \lambda_H &:= u_2(t) (\bar{\beta}_{HH} I_H + \bar{\beta}_{AH} I_A),
        \\
        \lambda_A &:= 
            u_2(t)(\bar{\beta}_{HA} + \bar{\beta}_{AA}) I_A
        + u_1(t)\beta_{AA} I_A,
        \\
        \frac{d S_H}{dt} &=
        b \left[S_A + (1 - q) I_A + R_A\right]
        -\lambda_H S_H - (\mu_H + m) S_H 
        - u_1(t)  S_H + \alpha_H u_1(t) R_H,
        \\
        \frac{d I_H}{dt} &=
        (1 - \sigma)  q  b  I_A
        + \lambda_H S_H - (\mu_H + \delta_H + \gamma_H)I_H
        - u_1(t) I_H,
        \\
        \frac{d R_H}{dt} &=
        \gamma_H I_H - (\mu_H + m) R_H
        + u_1(t) (S_H + I_H) - \alpha_H u_1(t) R_H,
        \\
        \frac{dS_A}{dt} &=
        \Lambda_A + m S_H - \lambda_A S_A - \mu_A S_A
        - u_1(t) S_A + \alpha_A u_1(t) R_A,
        \\
        \frac{d I_A}{dt} &=
        \lambda_A S_A - (\mu_A + \gamma_A)   I_A
        - u_1(t) I_A,
        \\
        \frac{dR_A}{dt} &=
        m R_H + \gamma_A I_A - \mu_A R_A
        + u_1(t) (S_A + I_A) - \alpha_A u_1(t) R_A.
    \end{aligned}
\end{equation}    
    

    \section{Existence and characterization of optimal policies}
    Our objective is minimize the cost function defined by
    \begin{equation}\label{cost_function}
        \begin{aligned}
        J(u_{1},u_{2}) = \int_{0}^{t_{f}}\left[a_{1}I_{H}(t) + a_{2}I_{A}(t) + b_{1}u_{1}(t)^2 + b_{2}u_{2}(t)^2\right]dt
        \end{aligned}
    \end{equation}
    
    We employ Pontryagin's Maximum Principle to converts~\eqref{controlled_system} and~\eqref{cost_function} into a problem of minimizing point-wise Hamiltonian $H$, with respect to $(u_{1},u_{2})$. First we formulate the Hamiltonian $H$ from the cost functional~\eqref{cost_function} and the governing dynamics~\eqref{controlled_system}.
    \begin{equation}
        \begin{aligned}
            H = f(t) + g(t)\cdot \lambda(t),
         \end{aligned}
    \end{equation}
    where $f(t)$ is the..., $g(t)$ is the right-side of system~\eqref{controlled_system}, and $\lambda(t) := \left( \lambda_{1}, \lambda_{2}, \lambda_{3}, \lambda_{4}, \lambda_{5}, \lambda_{6} \right)$. It is important to mention that $\lambda_{1}$, $\lambda_{2}$, $\lambda_{3}$, $\lambda_{4}$, $\lambda_{5}$ and $\lambda_{6}$ are the adjoin or co-state variables solutions of the following system:
    \begin{equation}\label{adjoint_system}
    \begin{aligned}
        \frac{d \lambda_{1}}{dt} &=
        -b (S_{A}(t) + (1 - q) I_{A}(t) + R_{A}(t) ) + (m + \mu_{H}) S_{H}(t) - \alpha_{H} R_{H}(t) u_{1}(t)\\
        &\quad+ S_{H}(t) u_{1}(t) + (\bar{\beta}_{AH} I_{A}(t) + \bar{\beta}_{HH} I_{H}(t)) S_{H}(t) u_{2}(t),
        \\
        \frac{d \lambda_{2}}{dt} &=
        -b q (1 - \sigma) I_{A}(t) - (\bar{\beta}_{AH} I_{A}(t) + \bar{\beta}_{HH} I_{H}(t)) S_{H}(t) u_{2}(t)  + I_{H}(t) u_{1}(t)\\
        &\quad + (\gamma_{H} + \delta_{H} + \mu_{H}) I_{H}(t),
        \\
        \frac{d \lambda_{3}}{dt} &=
        -\gamma_{H} I_{H}(t) + (m + \mu_{H}) R_{H}(t) + \alpha_{H} R_{H}(t) u_{1}(t) - (I_{H}(t) + S_{H}(t)) u_{1}(t),
        \\[0.3cm]
        \frac{d \lambda_{4}}{dt} &=
        -\Lambda_{A} - u_{1} \alpha_{A} R_{A}(t) + \mu_{A} S_{A}(t) - m S_{H}(t) + S_{A}(t) (\beta_{AA} I_{A}(t) u_{1}(t)\\
        &\quad + (\bar{\beta}_{AA} I_{A}(t) + \bar{\beta}_{HA} I_{H}(t)) u_{2}(t)) + S_{A}(t) u_{1}(t),
        \\
        \frac{d \lambda_{5}}{dt} &=
        (\gamma_{A} + \mu_{A}) I_{A}(t) + I_{A}(t) u_{1}(t) - S_{A}(t) (\beta_{AA} I_{A}(t) u_{1}(t) + (\bar{\beta}_{AA} I_{A}(t)\\
        &\quad + \bar{\beta}_{HA} I_{H}(t)) u_{2}(t)),
        \\
        \frac{d \lambda_{6}}{dt} &=
        -\gamma_{A} I_{A}(t) + \mu_{A} R_{A}(t) - m R_{H}(t) + \alpha_{A} R_{A}(t) u_{1}(t) - (I_{A}(t) + S_{A}(t)) u_{1}(t).
    \end{aligned}
\end{equation}

    The optimal conditions are given by solving the following equations by $u_{i}$. 
    \begin{equation*}
        \frac{\partial H}{\partial u_{i}} = 0,\quad i=1,2.
    \end{equation*}
    Thus,
    \begin{equation*}
        \begin{aligned}
        u_{1}^{*} &= \left(\frac{1}{2b_{1}}\right)
        \left[ 
            \lambda_{1}(t)
            \left(
            S_{H}(t) - \alpha_{H} R_{H}(t)
            \right)    
            + 
            \lambda_{2}(t) I_{H}(t)
            + 
            \lambda_{3}(t)
            \left(
            \alpha_{H} R_{H}(t) - S_{H}(t) - I_{H}(t)
            \right)
            \right.\\
            &\quad\quad\quad\quad \left.
            + 
            \lambda_{4}(t)
            \left(
                \beta_{AA} I_{A}(t) S_{A}(t) + S_{A}(t) - \alpha_{A}R_{A}(t)
            \right)
            + 
            \lambda_{5}(t)
            \left(
                I_{A}(t) - \beta_{AA} I_{A}(t) S_{A}(t)
            \right)
            \right.\\
            &\quad\quad\quad\quad \left.
            +
            \lambda_{6}(t)
            \left(
                \alpha_{A} R_{A}(t) - S_{A}(t) - I_{A}(t)
            \right)
        \right]\\[0.2cm]
        u_{2}^{*} &= \left(\frac{1}{2b_{2}}\right)
        \left[
            \left(
                \bar{\beta}_{AH} I_{A}(t) S_{H}(t)
                + 
                \bar{\beta}_{HH} I_{H}(t) S_{H}(t)
            \right)
            \left(
                \lambda_{1}(t) - \lambda_{2}(t)
            \right)
            \right.\\
            &\quad\quad\quad\quad\quad \left.
            +
            \left(
                \bar{\beta}_{AA} I_{A}(t) S_{A}(t)
                +
                \bar{\beta}_{HA} I_{H}(t) S_{A}(t)
            \right)
            \left(
                \lambda_{4}(t) - \lambda_{5}(t)
            \right)
        \right]
        \end{aligned}
    \end{equation*}
    \section{Numerical analysis}
    \section{Numerical experiments}
    \section{Concluding remarks}
    %%%%%%%%%%%%%%%%%%%%%%%%%%%%%%%%%%%%%%%%%%%%%%%%%%%%%
%    \pagebreak
%    \appendix
%    \renewcommand*{\thesection}{\appendixname~\Alph{section}}
%    \renewcommand{\thetheorem}{\Alph{section}.\arabic{theorem}}
%    \renewcommand{\thedefinition}{\Alph{section}.\arabic{definition}}
%    \input{Sections/Appendix/appendix}
    \section*{References}
    \bibliographystyle{model6-num-names}
    \bibliography{sde}
\end{document}
%
